\documentclass{article}
% --- General Packages ---
\usepackage[utf8]{inputenc}
\usepackage[T1]{fontenc}
\usepackage[english]{babel}
\usepackage{fullpage}
\usepackage{geometry}
\usepackage{graphicx}
\usepackage{hyperref}
\usepackage{url}
\usepackage{setspace}
\usepackage{comment}
\usepackage{xcolor}

% --- Math Packages ---
\usepackage{amsmath,amssymb,amsthm,nccmath}
\usepackage{amsfonts}
\usepackage{bm}

% --- Table & Figure Packages ---
\usepackage{booktabs}
\usepackage{multirow}
\usepackage{float}

% --- Code & Text Packages ---
\usepackage{listings}
\usepackage[autostyle, english = american]{csquotes}
\usepackage{algpseudocode}
\usepackage{tikz}
\usepackage{subcaption}
\usetikzlibrary {automata,positioning}

% --- Listing Style Configuration ---
\definecolor{codegreen}{rgb}{0,0.6,0}
\definecolor{codegray}{rgb}{0.5,0.5,0.5}
\definecolor{codepurple}{rgb}{0.58,0,0.82}
\definecolor{backcolour}{rgb}{0.95,0.95,0.92}

\lstdefinestyle{mystyle}{
    backgroundcolor=\color{backcolour},
    commentstyle=\color{codegreen},
    keywordstyle=\color{magenta},
    numberstyle=\tiny\color{codegray},
    stringstyle=\color{codepurple},
    basicstyle=\ttfamily\scriptsize,
    breakatwhitespace=false,
    breaklines=true,
    captionpos=b,
    keepspaces=true,
    numbers=left,
    numbersep=1pt,
    showspaces=false,
    showstringspaces=false,
    showtabs=false,
    tabsize=2,
    inputencoding=utf8,
    extendedchars=true,
}
\lstset{style=mystyle,
    literate={«}{{\guillemotleft}}1{»}{{\guillemotright}}1{’}{{\textquoteright}}1
}
\setlength{\parindent}{0in}
\setlength{\parindent}{0in}
\begin{document}
\begin{titlepage}
	\begin{center}
		\vspace*{1cm}

		\Huge
		\textbf{Devoir 2}

		\vspace{0.5cm}
		\LARGE

		\vspace{1.5cm}

        
		\textbf{Samuel Fournier}\\20218212 \\
		\vfill
		\textbf{Thierry Bédard-Cortey}\\20307759 \\
		\vfill


		Dans le cadre du cours\\
		IFT 6150


		\vspace{0.8cm}

		\includegraphics[width=0.4\textwidth]{udem.jpg}

		\Large
		Département d'informatique et de recherche opérationnelle\\
		Université de Montréal\\
		Canada\\
		29 octobre 2025

	\end{center}
\end{titlepage}

\section{Introduction}
% TODO: Contexte général et motivations.

\section{Contexte et problématique}
% TODO: Rôle de l'ACL Anthology et conférences cibles (ACL/NAACL/EACL/EMNLP/COLING/LREC/Findings).
% TODO: Taxonomie des tâches de classification (binaire, multi-classe, multi-label; domaines: sentiment, polarité, etc.).
% TODO: Notion de popularité (compte de citations) et enjeux pour la recherche.

\section{Méthodologie de cartographie}
\subsection{Collecte et filtrage des articles}
% TODO: Source (API/exports ACL), période, conférences, critères de filtrage "classification".
\subsection{Extraction d'attributs}
% TODO: Type de tâche, domaine, métriques reportées, nombre de benchmarks, citations, etc.
\subsection{Analyse descriptive}
% TODO: Statistiques et figures (ex.: histogrammes de types, évolution temporelle, parts par conférence).

\section{Sélection de tâches sous-exploitées}
% TODO: Critère d'exclusion: >200 citations interdit.
% TODO: Présentation brève des 2+ tâches retenues (données, labels, taille, splits, licences).
% TODO: Justification du choix (intérêt, lacunes, faisabilité).

\section{Approches expérimentales}
\subsection{Pipeline générique}
% TODO: Prétraitement, gestion des labels, métriques (accuracy, macro-F1, etc.), reproductibilité.
\subsection{Méthodes classiques (scikit-learn)}
% TODO: LR/SVM/RF; recherche d'hyperparamètres; validation croisée.
\subsection{Modèles avancés}
% TODO: BERT/LLM (zero/few-shot, fine-tuning, prompt-based); détails d'entraînement et ressources.
\subsection{Détails d'implémentation}
% TODO: Organisation du code, temps de calcul, matériel, seeds.

\section{Résultats et analyse}
\subsection{Tableaux de performance}
% TODO: Table(s) comparative(s) par tâche et par méthode.
\subsection{Comparaison aux références}
% TODO: Par rapport aux résultats des articles sources (si disponibles).
\subsection{Analyse critique}
% TODO: Forces/faiblesses, robustesse, erreurs typiques, complémentarités entre approches.

\section{Discussion}
% TODO: Leçons générales de la cartographie et des expériences.
% TODO: Impact de la rareté/popularité sur la performance et la recherche.
% TODO: Limites (données, métriques, budget de calcul) et menaces à la validité.

\section{Conclusion et travaux futurs}
% TODO: Synthèse des contributions et résultats.
% TODO: Pistes d'amélioration du pipeline et nouvelles tâches à explorer.

\begin{thebibliography}{99}
% TODO: Références aux articles introduisant les jeux de données étudiés (obligatoire).
% TODO: Références techniques (BERT, scikit-learn, etc.). Respecter un style cohérent.
\end{thebibliography}

\appendix
\section{Annexes (non comptées dans les 8 pages)}
% TODO: Détails complémentaires, figures supplémentaires, logs, configurations.
\end{document}